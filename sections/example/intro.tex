\section{Introduction}
This primer is primarily intended to provide guidance for writing academic and forensic reports in \LaTeX\ to anyone/anything wishing to do so in the context of the \texttt{uswdwms} package that it is written with and, at least in part, documents.

However, the techniques described in this primer can also be applied to other documents \textit{(such as penetration test reports)} in like manner and some additional commands and environments are provided for use in such documents by the \texttt{uswdwms} package.

This primer is intended to:
\begin{methodology0}
  \item Outline \texttt{uswdwms} package installation and usage
  \item Detail basic \LaTeX\ concepts in the context of the \texttt{uswdwms} package -- these include:
    \begin{itemize}
      \item document classes
      \item \LaTeX\ packages
      \item syntax -- basics, commands, environments, escapes, etc
      \item title pages
      \item headers and footers with \texttt{fancyhdr}
      \item sectioning
      \item table of contents
      \item labels and refs with \texttt{\textbackslash ref} and \texttt{\textbackslash href}
      \item hyperlinks with \texttt{\textbackslash hyperref}
      \item footnotes
      \item lists -- itemize and enumerate with \texttt{enumitem}
      \item tables with \texttt{tabularx}
      \item figures -- including graphics with captions and referenceable labels
      \item listings -- displaying code listings with \texttt{minted} and \texttt{pygments}
      \item lists of figures and listings
      \item appendices
      \item referencing and citations with \texttt{natbib} and \texttt{bibtex}
      \item \todo{documents in development}
    \end{itemize}
  \item Showcase how the primitives provided by the \texttt{uswdwms} package can be used to spice up reports
  \item Explain the construction of these primitives and describe how they can be altered in order to customise reports to a greater depth
  \item Provide instruction as to how Python can be used to automate the templating of \LaTeX\ code
\end{methodology0}

\subsection{Sectioning}
\label{primer:sec:concepts:sectioning}
Sectioning commands\footnote{\href{https://en.wikibooks.org/wiki/LaTeX/Document_Structure\#Sectioning_commands}{\faBook\ Wikibooks: \LaTeX{}/Document Structure: Sectioning commands}} can be used to split \LaTeX\ documents into \texttt{part}, \texttt{chapter}, \texttt{section}, \texttt{subsection}, \texttt{subsubsection}, \texttt{paragraph}, and \texttt{subparagraph} sections.

It should be noted, however, that some of these commands are not available for some document classes. For example, the \texttt{\textbackslash chapter} command is only available in \texttt{book} and \texttt{report} classes and is not available in the \texttt{article} class upon which this primer is built. For longer reports, such as dissertation and thesis reports, using the \texttt{report} class would be more appropriate.

Using sections as a guideline for splitting a large \texttt{.tex} file into multiple smaller \texttt{.tex} files can help to improve an author's ability to make changes to document structure quickly and easily \textit{(especially when combined with using version control, such as git)}. As an example, \texttt{concepts.tex} in the source code for this primer merely starts a \texttt{\textbackslash section} called ``Basic Concepts'' and then imports each concept with the \term{\textbackslash input\{...\}}. The \texttt{.tex} file for that specific concept then begins with a \texttt{\textbackslash subsection} command. This allows for the individual concept subsections to be reordered within the overall primer output by modifying only a couple of lines in \texttt{concepts.tex}, as opposed to having to cut-and-paste large sections of \LaTeX\ code \textit{(which would be required if every concept was detailed together in a singular file)}.

\todo{note about denumbering/hiding sections from the TOC with *}

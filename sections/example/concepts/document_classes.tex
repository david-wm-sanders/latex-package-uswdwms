\subsection{Document Classes}
The start of a top-level \LaTeX\ \texttt{.tex} file begins with a \term{\textbackslash documentclass[options]\{class\}} definition, as shown in Listing~\ref{example:lst:document_class:report}.

This command is used to configure options and set the document class. In the case of reports based on the \uswdwmspkg{}, it is used to set the default text font size \textit{(\texttt{[12pt]})} and the document class \textit{(\texttt{\{article\}})} by default only.

\begin{listing}[H]
  \captionsetup{skip=\skiplistingcaptionlen}
  \begin{minted}{tex}
    \documentclass[12pt]{article}
  \end{minted}
  \caption{\texttt{\textbackslash documentclass} command example}
  \label{example:lst:document_class:report}
\end{listing}

Other options can be set in the \texttt{\textbackslash documentclass} directive to customise the specified class. These include:

\begin{itemize}
  \item \texttt{draft} can be used to make \LaTeX\ show frames when images would be inserted and indicate hyphenation and justification issues in the output
  \item \texttt{twocolumn} can be used to instruct \LaTeX\ to write the document with two columns
  \item \texttt{twoside} can be used to specify that a double sided output document should be typeset - by default, the classes \texttt{article} and \texttt{report} are single-sided and the \texttt{book} class is double-sided.
\end{itemize}

Layout options, such as \texttt{a4paper} and \texttt{landscape}, could be used to customise the paper size and document layout. However, the \uswdwmspkg{} makes use of the \texttt{geometry} package for this task \textit{(shown in Listing~\ref{example:lst:document_class:usepackage_geometry})} as this package provides much more finessed options for sizing and laying out documents than the \texttt{\textbackslash documentclass} directive.

\begin{listing}[H]
  \captionsetup{skip=\skiplistingcaptionlen}
  \begin{minted}{tex}
    \usepackage[a4paper, margin=0.75in]{geometry}
  \end{minted}
  \caption{\texttt{geometry} package import and configuration example}
  \label{example:lst:document_class:usepackage_geometry}
\end{listing}

\LaTeX\ also provides some other classes, such as \texttt{report} \textit{(for longer report documents that contain multiple chapters)}, \texttt{letter} \textit{(for writing letters)}, and \texttt{beamer}\footnote{\href{https://en.wikibooks.org/wiki/LaTeX/Presentations}{Wikibooks: \LaTeX{}/Presentations}} \textit{(for creating presentation slides)} that can be used in lieu of the \texttt{article} class.

Further information and reading material on document classes can be found at: \href{https://en.wikibooks.org/wiki/LaTeX/Document_Structure#Document_classes}{Wikibooks: \LaTeX{}/Document Structure}.

\subsection{Basic \LaTeX\ syntax}
\subsubsection{\texttt{\textbackslash def}}
\term{\textbackslash def} is a \TeX\ primitive that can be used to define shortcuts for strings that can contain \LaTeX\ commands.

Listing~\ref{example:lst:basic_syntax:def_uswdwmspkg} shows how a shortcut for ``\uswdwmspkg{}'' is defined and how it can be used in two different ways depending on its position in a sentence. This will render:\\
A \uswdwmspkg\ primer example... in a report based on the \uswdwmspkg{}.

Listing~\ref{example:lst:basic_syntax:def_report_preamble} shows how \term{\textbackslash def} is used to configure variables for \LaTeX\ reports based on the \uswdwmspkg{}. These variables are then used to populate the headers and footers, and the table of contents.

Using \term{\textbackslash def} to define variables for things such as the title means that if changing the title of the report becomes necessary in the future then only one line of the \LaTeX\ document source needs to be modified -- the modification will then apply everywhere the shortcut is used.

\begin{listing}[H]
  \captionsetup{skip=\skiplistingcaptionlen}
  \begin{minted}{tex}
    % Define a shorthand for `\texttt{uswdwms} package`
    \def \uswdwmspkg{\texttt{uswdwms} package}
    A \uswdwmspkg\ primer example... in a report based on the \uswdwmspkg{}.
  \end{minted}
  \caption{\texttt{\textbackslash def \textbackslash uswdwmspkg} command example}
  \label{example:lst:basic_syntax:def_uswdwmspkg}
\end{listing}
\begin{listing}[H]
  \captionsetup{skip=\skiplistingcaptionlen}
  \begin{minted}{tex}
    % Define document variables
    \def \organisation{University of South Wales}
    \def \division{Applied Cyber Security}
    \def \reportcode{acs00001}
    \def \reportcategory{\LaTeX\ Reports}
    \def \reportaudience{Humanity\\AI}
    \title{Primer for Academic and Forensic Reports}
    \author{David Sanders}
    \def \authornick{dwms}
    \date{\today}
  \end{minted}
  \caption{\texttt{\textbackslash def} report preamble example}
  \label{example:lst:basic_syntax:def_report_preamble}
\end{listing}

\subsubsection{\texttt{\textbackslash let}}
\term{\textbackslash let}\footnote{\href{https://tex.stackexchange.com/questions/258/what-is-the-difference-between-let-and-def}{\faStackExchange\ What is the difference between \textbackslash let and \textbackslash def?}} is a \TeX\ primitive that allows the content of a command to be copied into a new command.

Listing~\ref{example:lst:basic_syntax:let} shows how \term{\textbackslash let} is used in \texttt{report.tex} to copy the contents set by the \texttt{\textbackslash title}, \texttt{\textbackslash author}, and \texttt{\textbackslash date} directives into separate variables that can be used without \term{\textbackslash makeatletter}.

It can also be very useful when using an existing command in a redefinition of the existing command is desired\footnote{\href{https://en.wikibooks.org/wiki/TeX/let}{\faBook\ Wikibooks: \TeX{}/\texttt{\textbackslash let}}}.

\begin{listing}[H]
  \captionsetup{skip=\skiplistingcaptionlen}
  \begin{minted}{tex}
    \makeatletter
    \let\thetitle\@title
    \let\theauthor\@author
    \let\thedate\@date
    \makeatother
  \end{minted}
  \caption{\texttt{\textbackslash let} command example}
  \label{example:lst:basic_syntax:let}
\end{listing}

\pagebreak
\subsubsection*{Commands}
\begin{listing}[H]
  \captionsetup{skip=\skiplistingcaptionlen}
  \begin{minted}{tex}
    % Set up an inline todo command
    \newcommand{\todo}[1]{\textcolor{red}{todo: #1}}
  \end{minted}
  \caption{\texttt{\textbackslash newcommand} command example}
  \label{example:lst:newcommand}
\end{listing}

\subsubsection*{Environments}
\begin{listing}[H]
  \captionsetup{skip=\skiplistingcaptionlen}
  \begin{minted}{tex}
    % Set up a todo environment
    \newenvironment{todoenv}
      {\color{red}todo:}
      {\color{black}}
  \end{minted}
  \caption{\texttt{\textbackslash newenvironment} command example}
  \label{example:lst:newenvironment}
\end{listing}
\todo{should note here \textbackslash NewEnviron is much nicer}

\subsubsection*{Escaping special characters}
\todo{some characters like \_, \&, \$, \^, \~\ require escaping... this needs to be mentioned}

\todo{also drop \textbackslash textbackslash (\textbackslash) and \textbackslash textasciitilde (\textasciitilde)}

\subsubsection*{Styling text in \textbf{bold}, \textit{italics}, \texttt{monospace}, and \textsc{Small Capitals}}
\todo{Talk about \textbf{bold stuff}, \textit{italic stuff}, \texttt{mono stuff}, etc}

\todo{Talk about \textcolor{nicer-blue}{colouring!}}
